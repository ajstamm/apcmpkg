\begin{titlepage}
\begin{center}

\pagestyle{empty}
% \pagenumbering{gobble}
% first numbered page should be ii - why?
\pagenumbering{roman}
\thispagestyle{empty}

\normalsize

Association of Prenatal Exposure to Air Pollutants with Select Birth Defects\\*[20pt]

Using the Case-Cohort Approach\\*[20pt]


by\\*[20pt]


Abigail Jeanne Stamm\\*[20pt]

\vfill


A Dissertation\\*[20pt]

Submitted to the University at Albany, State University of New York\\*[20pt]

In Partial Fulfillment of\\*[20pt]

The Requirements for the Degree of\\*[20pt]

Doctor of Philosophy\\*[30pt]

\vfill

School of Public Health\\*[20pt]

Department of Epidemiology and Biostatistics\\*[20pt]

2023
\end{center}

\pagebreak



% if I use &, I need to escape it



\section*{Abstract}

\doublespacing

\textbf{Background:} 
Annually, 3-5\% of infants are born with birth defects in the United States. In \gls{nys}, air pollution, specifically \gls{o3} and \gls{pm25}, is detectable at levels that affect human health. Air pollution is associated with poor birth outcomes and contains components that are suspected to cause oxidative stress, which is one mechanism by which birth defects are formed.  

\textbf{Methods:} 
This study investigated the relationship between the development of select birth defects (oral clefts, craniosynostosis, and clubfoot) in singleton live births in NYS between 2002 and 2015 and weekly peak and average concentrations of \gls{o3} and \gls{pm25} at the mother's residence using modeled air pollution data and single-pollutant distributed lag logistic models. The second analysis incorporated green space at various buffers using the \gls{nlcd}. The third analysis explored multi-pollutant models.  

\textbf{Results:} 
\gls{o3} most greatly affected risk of clubfoot around conception in the single-pollutant model and weeks 5 and 9 of pregnancy in the multi-pollutant model. \gls{pm25} most greatly affected risk of clubfoot around weeks 5 and 11 in the single-pollutant model and weeks 10-11 in the multi-pollutant model. Green space did not affect these results. 
\gls{o3} most greatly affected risk of cleft \lcp around conception and week 7 in the single-pollutant model and weeks 4 and 7 in the multi-pollutant model. \gls{pm25} most greatly affected risk of cleft \lcp around weeks 5 and 8 in the single-pollutant model and weeks 3 and 8 in the multi-pollutant model. Green space did not affect these results. 
\gls{o3} most greatly affected risk of cleft palate around conception and weeks 7 and 11 in the single-pollutant model and this did not change in the multi-pollutant model. \gls{pm25} most greatly affected risk of cleft palate in month 2 in the single-pollutant model and this did not change in the multi-pollutant model. Green space slightly altered the effect of \gls{o3}. 
\gls{o3} most greatly affected risk of craniosynostosis around weeks 7-8 in the single-pollutant model and this did not change in the multi-pollutant model. \gls{pm25} most greatly affected risk of cleft palate around conception and weeks 6-7 and 11 in the single-pollutant model and weeks 4 and 7 in the multi-pollutant model. Green space slightly altered the effect of \gls{pm}.   

\textbf{Conclusion:} 
\gls{o3} and \gls{pm25} affected the risk of all birth defects evaluated. In multi-pollutant models, these effects were altered for clubfoot, cleft \lcp, and craniosynostosis. Including green space altered these effects for cleft palate and craniosynostosis.

\pagebreak

\end{titlepage}